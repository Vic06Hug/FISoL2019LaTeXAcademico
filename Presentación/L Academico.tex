\documentclass{beamer}
\usepackage[spanish]{babel}
\usepackage[utf8]{inputenc}
\usepackage{default}
\usepackage{hyperref}
\usetheme{CambridgeUS}
\usepackage{color,xcolor}
\usepackage{float}
\usepackage{graphicx}
\usepackage{ragged2e}
\title{Taller académico de \LaTeX}
\author{Ponentes: Callejas Cipriano\\ Vázquez Víctor}
\institute{FLISoL\\ FES Acatlán, UNAM}
\date{\small{\today}}
\begin{document}
\frame{\titlepage}
\frame{\tableofcontents}
\section{MathJax}
\begin{frame}{¿Qué es MathJax\footnote{https://www.mathjax.org/}}
\justify
Es un motod e visualización de JavaScript de código abierto para la notación \LaTeX\, de MathML y ASCIIMATH que funciona en todos 
los navegadores modernos
\begin{center}
 \includegraphics[scale=0.5]{mathjax.png}
\end{center}


\end{frame}
\begin{frame}{Objetivo}
 \justify
Consilidar los recientes avances en tecnologías web en una única plataforma definitiva matemática en la web que soporta la 
mayoría de navegadores y hasta en dispositivos móviles
\end{frame}
\begin{frame}{¿Cómo funciona?}
 MathJax es modular, así que deja cargar componentes que sean solamente necesarios y ouede ampliarse más capacidad según las 
necesidades del usuario.
\end{frame}
\begin{frame}{Algunas características}
\justify
\begin{itemize}
 \item Es altamente \texttt{configurable} permitiendo a los autores personalizarlo para los requisitos especiales de sus sitios 
web
\item Tiene una interfaz de programación de aplicaciones (API) enriquecida que se puede utilizar para hacer las fórmulas 
matemáticas en sus págias web para que sean \textbf{interactivas y dinámicas}
\end{itemize}
\end{frame}
\section{Moodle}
\begin{frame}{¿Qué es?\footnote{https://moodle.org/}}
\justify
Moodle es una plataforma de aprendizaje de \texttt{código abierto}, diseñada para proporcionarle a educadores, administradores y 
estudiantes un \textunderscore{sistema integrado único y robusto} para crear Ambientes Virtuales de Aprendizaje (AVAs)
personalizados.
\begin{center}
 \includegraphics[scale=0.5]{moodle.jpeg}
\end{center}
\end{frame}
\section{¿Porqué Moodle es de código abierto?}
\begin{center}
\begin{itemize}
\item Moodle es proporcionado gratuitamente como programa de \textit{Código abierto} bajo la licencia pública general 
\textit{(GNU)}.
\item Cualquier persona puede adaptar, extender o modificar Moodle, tanto para proyectos comerciales como no comerciales, sin 
pago de coutas por licenciamiento 
\end{itemize}
\end{center}
\end{document}
